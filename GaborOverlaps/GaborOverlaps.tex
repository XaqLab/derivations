\documentclass[12pt]{article}
%\usepackage{times}
\usepackage{color}
\usepackage{multirow}
\usepackage[tight,footnotesize]{subfigure}
\usepackage{sectsty}
\usepackage{titlesec}
\usepackage{authblk}
\usepackage[cm]{fullpage}
\usepackage{latexsym, bm, amsmath, amssymb, amsthm, bbm, epsf, graphicx}
\usepackage{import}
\usepackage{float}
\usepackage{color}
\usepackage{cite}
\usepackage{array}
\usepackage{mdwmath}
\usepackage{mdwtab}
\usepackage{eqparbox}
\usepackage[tight,footnotesize]{subfigure}
\usepackage{rotating}
\usepackage{bbm}
\usepackage{latexsym}


\import{/Users/xaq/Code/TeX/}{TeX_definitions.tex}




\title{\bf Gabor overlaps}
\author[1,2]{Xaq Pitkow}
\affil[1]{Rice University, Department of Electrical and Computer Engineering}
\affil[2]{Baylor College of Medicine, Department of Neuroscience}
\date{}

\begin{document}

\maketitle



This note computes the overlap between two Gabor functions. Gabors are defined as a sinusoid modulated by a gaussian envelope:
\be
g(\vx|\vmu,\sigma^2,\vk,\phi)=\exp{\left(-\frac{|\vx-\vmu|^2}{2\sigma^2}\right)}\cos{\left(\vk\cdot(\vx-\vmu)+\phi\right)}
\ee
We often want to compute the overlap between two Gabor functions,
\be
G_{mn}=\int d\vx\,g_m(\vx)g_n(\vx)
\label{eq:overlap}
\ee
where each Gabor has its own parameters $g_m(\vx)=g(\vx|\vmu_m,\sigma^2_m,\vk_m,\phi_m)$. For example, one Gabor might be a linear receptive field and the other might be a visual image.

To compute this integral, it is convenient to use complex exponentials instead of sines and cosines, $\cos{t}=\Re[e^{it}]=\frac{1}{2}(e^{it}+e^{-it})$. The real part of a product of complex exponentials is
\begin{align}
\Re[u v]&=\frac{1}{4}(u+\bar{u})(v+\bar{v})\\
&=\frac{1}{4}(uv+\bar{u}v+u\bar{v}+\bar{u}\bar{v})\\
&=\frac{1}{4}((uv+\overline{uv})+(u\bar{v}+\overline{u\bar{v}})\\
&=\frac{1}{2}(\Re[uv]+\Re[u\bar{v}])
\label{eq:productcomplex}
\end{align}
This means that to compute the overlap (\ref{eq:overlap}), we can define a complex Gabor
\be
h(\vx|\vmu,\sigma^2,\vk,\phi)=\exp{\left(-\frac{|\vx-\vmu|^2}{2\sigma^2}+i\left(\vk\cdot(\vx-\vmu)+\phi\right)\right)}
\ee
and then compute the complex overlaps
\be
H_{mn}=\int d\vx\,h_m(\vx)h_n(\vx)
\ee
Using this overlap with (\ref{eq:productcomplex}), we can compute
\be
G_{mn}=\tfrac{1}{2}(\Re[H_{mn}]+\Re[H_{m\bar{n}}])
\ee
where $H_{m\bar{n}}=\int d\vx\,h_m(\vx)\bar{h}_n(\vx)$, and the only difference between $h$ and $\bar{h}$ is that $\vk\to -\vk$ and $\phi\to -\phi$. Thus once we have one general $H_{mn}$, we can transform it to compute the target integral $G_{mn}$.

To compute $H_{mn}$, notice that the product of two complex gabors is yet another complex gabor,
\begin{align}
h_m(\vx)h_n(\vx) &= \exp{\left(-\frac{|\vx-\vmu_m|^2}{2\sigma_m^2}-\frac{|\vx-\vmu_n|^2}{2\sigma_n^2}+i\left(\vk_m\cdot(\vx-\vmu_m)+\phi_m+\vk_n\cdot(\vx-\vmu_n)+\phi_n\right)\right)}\\
&=\exp{\left(-\frac{1}{2}a |\vx|^2+\vb\cdot\vx+c\right)}
\label{eq:basicform}
\end{align}
where we have defined
\begin{align}
a&=\frac{1}{\sigma_m^2}+\frac{1}{\sigma_n^2}\\
\vb&=\frac{\vmu_m}{\sigma_m^2}+\frac{\vmu_n}{\sigma_n^2}+i\vk_m+i\vk_n\\
c&=-\frac{|\vmu_m|^2}{2\sigma_m^2}-\frac{|\vmu_n|^2}{2\sigma_n^2}+i\phi_m-i\vk_m\cdot\vmu_m+i\phi_n-i\vk_n\cdot\vmu_n
\end{align}
We can complete the square in form (\ref{eq:basicform}),
\be
h_m(\vx)h_n(\vx) = \exp{\left(-\frac{1}{2}a |\vx-\vb/a|^2\right)}\exp{\left(c+\frac{|\vb|^2}{2a}\right)}
\ee
and this form can be readily integrated to give
\be
H_{mn}=\int d\vx\, h_m(\vx)h_n(\vx)=\frac{2\pi}{a}\exp{\left(c+\frac{|\vb|^2}{2a}\right)}
\ee
where $|\vb|^2=\vb\cdot\vb$. Note that this magnitude is a {\it vector} magnitude, and {\it not} a complex magnitude $\vb\cdot\bar{\vb}$, so it can therefore still have an imaginary part.

Now we have to unpack the variables $a,\vb,c$. The magnitude $|\vb|^2$ is
\be
|\vb|^2=\left|\frac{\vmu_m}{\sigma_m^2}+\frac{\vmu_n}{\sigma_n^2}\right|^2-\left|\vk_m+\vk_n\right|^2+2i \left(\frac{\vmu_m}{\sigma_m^2}+\frac{\vmu_n}{\sigma_n^2}\right)\cdot\left(\vk_m+\vk_n\right)
\ee
Separating the argument of the exponential into real and imaginary parts, we find
\begin{align}
c+\frac{|\vb|^2}{2a}=&\left[-\frac{|\vmu_m|^2}{2\sigma_m^2}-\frac{|\vmu_n|^2}{2\sigma_n^2}
+\frac{1}{2}\left(\left|\frac{\vmu_m}{\sigma_m^2}+\frac{\vmu_n}{\sigma_n^2}\right|^2
-\left|\vk_m+\vk_n\right|^2\right)\left(\frac{1}{\sigma_m^2}+\frac{1}{\sigma_n^2}\right)^{-1}\right]\\
&+i\left[\left(\phi_m-\vk_m\cdot\vmu_m\right)+\left(\phi_n-\vk_n\cdot\vmu_n\right)+\left(\frac{\vmu_m}{\sigma_m^2}+\frac{\vmu_n}{\sigma_n^2}\right)\cdot\left(\vk_m+\vk_n\right)\left(\frac{1}{\sigma_m^2}+\frac{1}{\sigma_n^2}\right)^{-1}\right]\\
=&s+it
\end{align}
Using the real part of the exponential involving this term, we have
\begin{align}
\Re[H_{mn}]=&\frac{2\pi}{a}e^s\cos{t}\\
=&\frac{2\pi}{\sigma_m^{-2}+\sigma_n^{-2}}\exp{\left[-\frac{|\vmu_m|^2}{2\sigma_m^2}-\frac{|\vmu_n|^2}{2\sigma_n^2}
+\frac{1}{2}\left(\left|\frac{\vmu_m}{\sigma_m^2}+\frac{\vmu_n}{\sigma_n^2}\right|^2
-\left|\vk_m+\vk_n\right|^2\right)\left(\frac{1}{\sigma_m^2}+\frac{1}{\sigma_n^2}\right)^{-1}\right]}\\
&\cdot\cos{\left[\left(\phi_m-\vk_m\cdot\vmu_m\right)+\left(\phi_n-\vk_n\cdot\vmu_n\right)+\left(\frac{\vmu_m}{\sigma_m^2}+\frac{\vmu_n}{\sigma_n^2}\right)\cdot\left(\vk_m+\vk_n\right)\left(\frac{1}{\sigma_m^2}+\frac{1}{\sigma_n^2}\right)^{-1}\right]}
\end{align}
The same thing holds for the term $H_{m\bar{n}}$ but with $\vk_n\to-\vk_n$ and $\phi_n\to-\phi_n$, giving
\begin{align}
G_{mn}=&\tfrac{1}{2}(\Re[H_{mn}]+\Re[H_{mn}])\\
=&\frac{\pi}{\sigma_m^{-2}+\sigma_n^{-2}}\exp{\left[-\frac{|\vmu_m|^2}{2\sigma_m^2}-\frac{|\vmu_n|^2}{2\sigma_n^2}
+\frac{1}{2}\left|\frac{\vmu_m}{\sigma_m^2}+\frac{\vmu_n}{\sigma_n^2}\right|^2 \left(\frac{1}{\sigma_m^2}+\frac{1}{\sigma_n^2}\right)^{-1}\right]}\\
&\left(
\exp{\left[
-\tfrac{1}{2}\frac{\left|\vk_m{\color{red}+}\vk_n\right|^2}{\sigma_m^{-2}+\sigma_n^{-2}}
\right]}
\cos{\left[
\left(\phi_m-\vk_m\cdot\vmu_m\right){\color{red}+}\left(\phi_n-\vk_n\cdot\vmu_n\right)+\frac{\left(\frac{\vmu_m}{\sigma_m^2}+\frac{\vmu_n}{\sigma_n^2}\right)\cdot\left(\vk_m{\color{red}+}\vk_n\right)}
{\sigma_m^{-2}+\sigma_n^{-2}}
\right]}
\right.\\
&\left.
+\exp{\left[
-\tfrac{1}{2}\frac{\left|\vk_m{\color{red}-}\vk_n\right|^2}{\sigma_m^{-2}+\sigma_n^{-2}}
\right]}
\cos{\left[
\left(\phi_m-\vk_m\cdot\vmu_m\right){\color{red}-}\left(\phi_n-\vk_n\cdot\vmu_n\right)+\frac{\left(\frac{\vmu_m}{\sigma_m^2}+\frac{\vmu_n}{\sigma_n^2}\right)\cdot\left(\vk_m{\color{red}-}\vk_n\right)}
{\sigma_m^{-2}+\sigma_n^{-2}}
\right]}
\right)
\end{align}
where changes between the two terms have been highlighted in color.

I have checked this derivation against numerical integration and the two quantities agree.





\end{document}


